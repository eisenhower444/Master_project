\chapter{Conclusions and Further Work}

\noindent We conclude this thesis with a brief summary. We first discussed the recent advances which have allowed Deep Learning techniques to engender a string successes in pattern recognition, and particularly in the context of images. These successes have led to an increasing interest in applying these techniques to Medical Imaging. In the second chapter, we presented some background material on CNNs, covering convolutional layers, subsampling layers and an overview of the architectural organisation. In Chapter 3, we turn our attention to implementing a CNN ourselves using a tri-planar approach to generate the input set as a way of providing 3D information at a much lower cost than 3D patches along with a number of other practical considerations. We then performed model selection, finding a much better architecture and set of learning parameters than was initially proposed. To that end, we varied the number of convolutional and connected layers, the number of feature maps and hidden units, the type of activation function, the learning rate and the momentum. We then tried a different sampling procedure to increase the frequency of examples near the boundary of the atrium where most of the errors lie, yielding a significant increase in accuracy. Our final model gives a mean Dice coefficient of 0.985 across our 7 test CT scans. Most of the errors were at the border regions with the atrium where we expect them to be. However, despite the high rate of accuracy, the sensitivity was shown to be variable and some masks demonstrated errors far away from the atrium which we would expect the classifier not to commit. As a result, we do not expect this particular classifier to be useful for practicing radiologists in their work.\\

\noindent A number of things could be undertaken to improve the results. A first obvious step would be to increase the number of training examples. Another one would be to further improve with the sampling procedure in order to increase the number of training examples at the border regions. One could, for instance, systematically include all the pixels located at the border region of the atrium from each CT scan allocated to generating the training set. Another avenue for improvement would be to add more inputs channels to provide more information to the network, such as 3D patches.